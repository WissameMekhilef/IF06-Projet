\documentclass{beamer}

\usepackage[utf8]{inputenc}
\usepackage{default}
\usepackage[french]{babel}

\usetheme{Warsaw}

\title{Jeu de Taquin}
\subtitle{Manuel de l'utilisateur}
\author{ARNOULT Simon, MEKHILEF Wissame, OUSSAD Jihad \& RETY Martin}
\institute{Université d'Orléans}
\date{\today}

\begin{document}

  \begin{frame}
   \titlepage
  \end{frame}

  \begin{frame}
   \frametitle{Option jar}
	Voici la liste des différentes options jar:
	\begin{itemize}
	\item -name : Affiche le nom des développeurs
	\item -h : Affiche les différentes options
	\item -sol fichier.taq -j : Test si le fichier placé en paramètre à une solution
	\item -joue fichier.taq : Permet de jouer avec le fichier en paramètre
	\item -cal delai algo fichier.taq : Calcul une solution en utilisant l'algo en paramètre et d'un delai d'éxécution
	\item -anime : Calcul une solution comme -cal et affiche une animation des coups à faire
	\item -stat delai algo ficher.taq: Affiche les statistiques de résolution du taquin placé en paramètre
	\item -stat delai fichier.taq : Renvoie toutes les statistiques das un ficher html
	\item -alea n largeur hauteur delai fichier.taq : Applique ous les algo n fois sur des taquins de taille indiqué n paramètre
	\end{itemize}
   \end{frame}
   
  \begin{frame}
	\frametitle{Délai et algorithme}
	Lors de l'utilisation des options définies précédemment, le délai est exprimé en ms. Vous trouverez ci-dessous la liste des différents algorithmes utilisables:
	\begin{itemize}
	\item -algo manhattan
	\item -algo pile
	\item -algo file
	\end{itemize}
  \end{frame}
  \begin{frame}
   \frametitle{Exemples}
	Voici des exemples pour chaque option:
\begin{itemize}
	\item -name : java -jar taquin.jar -name
	\item -h : java -jar taquin.jar -h
	\item -sol fichier.taq -j : java -jar taquin.jar -sol taquin/taq1.taq -j
	\item -joue fichier.taq : java -jar taquin.jar -joue taquin/taq1.taq
	\item -cal delai algo fichier.taq : java -jar taquin.jar -cal 1000 -algo manhattan taquin/taq1.taq
	\item -anime delai algo ficher.taq : java-jar taquin.jar 1000 -algo manhattan taquin/taq1.taq
	\item -stat delai algo ficher.taq : java -jar taquin.jar -stat 1000 -algo manhattan taquin/taq1.taq
	\item -stat delai fichier.taq : java -jar taquin.jar -stat 1000 taquin/taq1.taq
	\item -alea n largeur hauteur delai fichier.taq : java -jar taquin.jar -alea 3 4 4 taquin/taq1.taq
	\end{itemize}
  \end{frame}
  \begin{frame}
   \frametitle{Jouer à une partie}
	Lors d'un jeu, à partir de l'option -joue, les différentes touches sont:
	\begin{itemize}
	\item z: déplacement de la case 0 vers le haut
	\item q: déplacement de la case 0 vers la gauche
	\item s: déplacement de la case 0 vers le bas
	\item d: déplacement de la case 0 vers la droite
	\end{itemize}
	\end{frame}
\end{document}
