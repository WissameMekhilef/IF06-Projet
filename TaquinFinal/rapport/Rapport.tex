\documentclass[a4paper,twoside,12pt]{report}
\usepackage[utf8x]{inputenc}
\usepackage{color}
\usepackage{fancyvrb} % pour mettre les verbatim dans des boites
%\usepackage[dvipdf]{graphics}
\usepackage[pdftex]{graphicx}
\usepackage{fancyhdr}
\usepackage{listings}
\usepackage{float}
\usepackage{listingsutf8}
\usepackage[T1]{fontenc}
\usepackage[french]{babel}
\usepackage{textcomp}
\usepackage[toc,page]{appendix}
\usepackage{hyperref}
% Quelques réglages particuliers
\oddsidemargin=-0.1in %xx
\evensidemargin=-0.1in
\textwidth=6.1in
\topmargin=-0.5in
\textheight=8.7in
\parskip 0.25in
\lstset{
language=Java,
basicstyle=\normalsize,
upquote=true,
aboveskip={1.5\baselineskip},
columns=fullflexible,
showstringspaces=false,
extendedchars=true,
breaklines=true,
showtabs=false,
showspaces=false,
showstringspaces=false,
identifierstyle=\ttfamily,
keywordstyle=\color[rgb]{1,0,1},
commentstyle=\color[rgb]{0.133,0.545,0.133},
stringstyle=\color[rgb]{0.627,0.126,0.941},
}
\fancyhead[RO,RE]{\includegraphics[width=0.5in]{Logo-univ-orleans.png}}
\fancyhead[CO,CE]{Rapport de projet}
\fancyhead[LO,LE]{Jeu de Taquin}
\fancyfoot[C]{2015}
\fancyfoot[RO, LE] {\thepage}
%\fancyfoot[LO, RE] { ARNOULT S., MEKHILEF W., OUSSAD J., RETY M.}
%
%Quelques macros
\newcommand{\moncode}[1]{\begin{center}
                        \lstinputlisting[inputencoding=utf8/latin1]{#1}
                        \end{center}}

\newcommand{\monimage}[4]{
\par\noindent
\begin{figure}[H] %on ouvre l'environnement figure
\begin{center}
\includegraphics[width=#4\textwidth]{#1} %ou image.png, .jpeg etc.
\caption{#2} %la légende
\label{#3} %l'étiquette pour faire référence à cette image
\end{center}
\end{figure} %on ferme l'environnement figure
}

\newcommand{\ml}[0]{\par\noindent}

%\hyphenation{suppléentaires}{sup-pl\'e-men-tai-res}
%opening
\title{\textcolor{blue}{\Large Rapport du projet IF06}\\\textcolor{blue}{\Large 
Jeu de Taquin}}
\vskip 1cm
\author{ARNOULT Simon, MEKHILEF Wissame, OUSSAD Jihad, RETY Martin 
\date{\today}}


\begin{document}


%
\begin{figure}[H]
\includegraphics[width=0.2\linewidth]{Logo-univ-orleans.png}
% \hfill

\end{figure}
\vspace{2cm}
%
\begin{center}
{\Huge Licence 2 Informatique\\\ \\Rapport du projet IF06}
\par\vspace{1.4cm}

{\Huge\bf \textcolor{red}{\bf Jeu de Taquin}}
\par\vspace{1.6cm}

{\Large       Réalisé par:}
\par\vspace{1.3cm}
{\large\bf \textcolor{blue}{ARNOULT Simon, MEKHILEF Wissame, OUSSAD Jihad, RETY 
Martin}}
\vfill
\today
\end{center}
%\newpage
\pagestyle{fancy}

\begin{abstract}
%
Nous avons le plaisir de vous présenter notre travail sur ce projet. Durant la 
deuxième année de la l2 Informatique
à l'Université d'Orléans, nous avons travaillé en groupe de quatre sur un projet 
de résolution de Taquin.
\end{abstract}
 
\newpage
\tableofcontents
\listoffigures
\newpage

\chapter{Introduction}
Nous allons ensemble, aborder quatres points principaux dans ce rapport:
\begin{itemize}
\item L'étude du projet
\item L'organisation de travail au sain du groupe
\item Le développement du code
\item L'analyse des algorithmes.
\end{itemize}
%
\chapter{Etude}
\par
L'étude du projet fut une première étape importante pour se mettre dans une 
bonne dynamique de groupe. Dès la première semaine nous avons réalisé une 
version jouable du Taquin, puis nous avons travaillé sur cette version durant 
toute la première phase d'essais.
\par\noindent
Rapidement il nous a fallu faire des essais sur ce jeu pour qu'il puisse se 
résoudre algorithmiquement. Nous avons chacun travaillé de notre coté sur nos 
idées pour pouvoir en tester un plus grand nombre, mais tout en restant en 
contact réguliérement pour avancer ensemble.
\par\noindent
Cette manière de répartir les tâches nous a permis de se rendre compte des 
difficultés que l'on allait rencontrer plus rapidement.
%
\section{Analyse de faisabilité}
\par
L'analyse de faisabilité fut un moment qui a duré du début du projet jusqu'aux 
vacances de février, durant cette phase chacun travaillant sur des 
fonctionnalités différentes, nous avons pu voir ou étaient les problèmes dans 
notre architecture de départ.
\par\noindent
Ces tests ont été très divers. Nous avons travaillé sur les fonctionnalités 
VT100 du terminal et la récupération des touches tapées par l'utilisateur. Mais 
nous avons aussi créé des algorithmes pour essayer de résoudre le Taquin.
\par\noindent
Tous ces tests nous ont permis au mois de février d'avoir une idée claire de 
l'architecture du projet.
\section{Conception UML}
\par
Cette analyse fin février a permis d'aboutir au diagramme UML suivant, diagramme 
qui n'a pas beaucoup évolué jusqu'à la version finale. Pour
faciliter la lecture nous avons divisé l'architecture en packages, seulement 3 
des packages sont présentés.
\ml
\monimage{../dev/jeuPackage.png}{JeuP}{Représentation UML du package jeu}{0.9}
\ml
\monimage{../dev/algoPackage.png}{AlgoP}{Représentation UML du package 
algo}{0.9}
\ml
\monimage{../dev/automatePackage.png}{AutomateP}{Représentation UML du package 
automate}{0.3}
\chapter{La gestion de projet}
%
\par
La gestion de projet est à la base de tout projet, et encore plus quand il se 
fait avec une équipe de quatre personnes.
\section{Le travail de groupe}
%
\par
La gestion de projet fut au coeur de nos préoccupations avant même que le projet 
ne démarre, nous avons cherché à créer une équipe dynamique. Nous nous sommes 
donc vus régulièrement dans les salles de l'Université. Malgré celà il nous a 
fallu mettre en place des moyens dédiés pour faciliter le travail et éviter une 
dégradation de l'entente.
\par\noindent
 Dès la première semaine, nous avons mis en place un dépôt git sur le site de 
l'hébergeur GitHub, ce dépôt nous a permis d'avoir le réflexe de l'utiliser même 
 si nous avons rencontré quelques problèmes, il nous a permis à chacun de voir 
l'avancement du projet. Le dépôt est disponible à cette adresse : https://github.com/wissame95/IF06-Projet.git.
\monimage{DepotGit.png}{Dépôt sur GitHub}{depot}{0.8}
\par\noindent
 Cependant un tel dépôt ne répond pas à la question de la communication, chacun 
habitant une ville différente, nous avons donc communiqué via Skype pour 
remédier à ce problème.

\chapter{Phase de développement}
%
\par
Pour la phase de développement nous avons chacun de notre côté développé une 
partie de l'application. Nous nous sommes partagés les différents 
algorithmes, mais aussi les autres fontionnalités comme les tests Junit.
\section{L'architecture}
\par
Nous avons fait le choix de répartir dans différents packages les classes et 
interfaces. Nous avons eu des choix à faire à plusieurs niveaux. Par
exemple un choix simple pour la grille de jeu, nous avons choisi une matrice 
d'entier en remplacement d'une ArrayList d'entiers. Mais aussi des choix plus
compliqués, comme la représentation d'un sommet qui dans notre cas est la classe 
Taquin.
\par\noindent
L'architecture se décompose en 7 packages : 
\begin{description}
 \item [algo] Contient les interfaces EnsembleATraiter et EnsembleMarque, toutes 
les classes implémentants ces interfaces et une classe Algo.
 \item [jeu] Contient une interface Jeu, la classe Taquin. Mais aussi les 
commandes, les actions, et les positions finales.
 \item [comparateur] Contient deux comparateurs Manhattan et DepthManhattan.
 \item [exceptions] Contient toutes les exceptions que nous avons dû créer.
 \item [automate] Contient les classes relatives à l'automate.
 \item [junit] Contient les différents tests JUnit.
 \item [main] Contient une seule classe, Main. Elle gère la lecture des 
paramètres et l'execution des méthodes dans les autres classes, c'est
 le lien entre l'utilisateur et l'application.
\end{description}

\section{Les fichiers test}
\par
Rapidement pour vérifier le fonctionnement du programme nous avons dû créer des fichiers tests de taquins témoins. Ces fichiers se trouvent dans le dossier taquin
à la racine du projet. Nous en avons utilisés un petit nombre, avec des tailles différentes et ``bien mélangés''.
%
\section{Fonction non implémenté}
Nous avons rencontrer des soucis sur l'implémentation de la pile d'action, le problème intervient au moment de remonter les parents d'un Taquin, il vient
certainement d'un null. Ceci a donc engendrer le fait de ne pas pouvoir tester nos méthodes sur le parcours progressif.
Enfin, nous avons aussi rencontrer des soucis de lecture des Junit et benchmarks dans une classe interne.
\chapter{Analyse et Conclusion}
%
\section{Analyse des Benchmark}
%
\par
Les benchmarks fournis par le module h2, nous ont permis d'améliorer l'efficacité de l'application, mais aussi de montrer le bon fonctionnement des méthodes.
Nous avons donc pour Algo et pour Taquin créés deux classes, une prouvant que le programme tourne correctement et l'autre pour montrer la rapidité d'éxécution
avec un problème de plus en plus grand. Nous avons aussi écrit un test pour l'ensemble incomplet pour observer la résolution et le temps nécessaire à celle-ci, en
fonction de la taille de l'ensemble utilisé.
%
\subsection{Efficacité des fonctions du Taquin}
%
Ci-dessous vous pouvez voir un récapitulatif des fonctions coûteuses de la classe Taquin.
\monimage{taquinSpeed.png}{Efficacité des fonctions du Taquin}{Eff}{0.6}
%
\subsection{Efficacité des algorithmes}
Les tests suivants ont été réalisés à l'aide de taq1.taq qui est un taquin de taille 3x3. On peut observer les différentes vitesse de résolution pour le même Taquin en fonction de
l'algorithme.
\ml
Les algorithmes utilisant un ensemble complet sont plus lent que les algortihmes utilisant un ensemble incomplet. De meme on observe que les tas sont plus rapide que les
files et piles, qui demande le plus de temps de calcule.
\ml
Les algorithmes basées sur un automate sont les plus efficace, cependant l'efficacité dépend de la profondeur de l'apprentissage de l'automate et de la taille du taquin.
\ml
 En effet, le nombre de coup redondant grandit en même temps que la taille du taquin, il est donc préférable d'utiliser cette automate sur un taquin de grande taille.
\monimage{algoSpeed.png}{Diagramme des temps d'execution}{tempsExec}{0.6}

\subsection{Etude sur l'ensemble incomplet}
Ci-dessous vous pouvez voir un diagramme montrant le temps nécessaire à la résolution d'un taquin avec un ensemble incomplet sur 5 tailles différentes.
\monimage{ensembleincomplettest2.pdf}{Exemple}{EX}{0.6}

\section{Conclusion}
%
En définitive, ce projet nous a permis d'acquérir une certaine autonomie, puisque nos simples connaissances ne suffisaient pas à implémenter efficacement tous les aspects du projet.
Nous avons également dû faire preuve de rigueur et d'inventivité afin de contourner certaines difficultés inhérentes à la complexité algorithmique du projet.
\chapter{Resources utilisées}
\begin{itemize}
 \item JUNIT
 \item JAVA
 \item Dia
 \item Eclipse
 \item Argouml
 \item \LaTeX (sous kile)
 \item Google Drive
 \item Kate
 \item Git, GitHub
\end{itemize}

\end{document}
