\documentclass[a4paper,twoside,12pt]{report}
\usepackage[utf8x]{inputenc}
\usepackage{color}
\usepackage{fancyvrb} % pour mettre les verbatim dans des boites
%\usepackage[dvipdf]{graphics}
\usepackage[pdftex]{graphicx}
\usepackage{fancyhdr}
\usepackage{listings}
\usepackage{float}
\usepackage{listingsutf8}
\usepackage[T1]{fontenc}
\usepackage[french]{babel}
\usepackage{textcomp}
\usepackage[toc,page]{appendix}
%\usepackage{hyperref}
% Quelques réglages particuliers
\oddsidemargin=-0.1in %xx
\evensidemargin=-0.1in
\textwidth=6.1in
\topmargin=-0.5in
\textheight=8.7in
\parskip 0.25in
\lstset{
language=Java,
basicstyle=\normalsize,
upquote=true,
aboveskip={1.5\baselineskip},
columns=fullflexible,
showstringspaces=false,
extendedchars=true,
breaklines=true,
showtabs=false,
showspaces=false,
showstringspaces=false,
identifierstyle=\ttfamily,
keywordstyle=\color[rgb]{1,0,1},
commentstyle=\color[rgb]{0.133,0.545,0.133},
stringstyle=\color[rgb]{0.627,0.126,0.941},
}
\fancyhead[RO,RE]{\includegraphics[width=0.5in]{Logo-univ-orleans.png}}
\fancyhead[CO,CE]{Rapport de projet}
\fancyhead[LO,LE]{Jeu de Taquin}
\fancyfoot[C]{2014}
\fancyfoot[RO, LE] {\thepage}
%\fancyfoot[LO, RE] { ARNOULT S., MEKHILEF W., OUSSAD J., RETY M.}
%
%Quelques macros
\newcommand{\moncode}[1]{\begin{center}
                        \lstinputlisting[inputencoding=utf8/latin1]{#1}
                        \end{center}}

\newcommand{\monimage}[3]{
\par\noindent
\begin{figure}[H] %on ouvre l'environnement figure
\begin{center}
\includegraphics[width=0.7\textwidth]{#1} %ou image.png, .jpeg etc.
\caption{#2} %la légende
\label{#3} %l'étiquette pour faire référence à cette image
\end{center}
\end{figure} %on ferme l'environnement figure
}

\newcommand{\ml}[0]{\par\noindent}

%\hyphenation{supplémentaires}{sup-pl\'e-men-tai-res}
%opening
\title{\textcolor{blue}{\Large Rapport du projet IF06}\\\textcolor{blue}{\Large Jeu de Taquin}}
\vskip 1cm
\author{ARNOULT Simon, MEKHILEF Wissame, OUSSAD Jihad, RETY Martin \date{\today}}


\begin{document}

\thispagestyle{empty}
%
\begin{figure}[H]
\includegraphics[width=0.2\linewidth]{Logo-univ-orleans.png}
% \hfill

\end{figure}
\vspace{2cm}
%
\begin{center}
{\Huge Licence 2 Informatique\\\ \\Rapport du projet IF04}
\par\vspace{1.4cm}

{\Huge\bf \textcolor{red}{\bf Jeu de Taquin}}
\par\vspace{1.6cm}

{\Large       Réalisé par:}
\par\vspace{1.3cm}
{\large\bf \textcolor{blue}{ARNOULT Simon, MEKHILEF Wissame, OUSSAD Jihad, RETY Martin}}
\vfill
\today
\end{center}
\newpage
\pagestyle{fancy}

\begin{abstract}
%
Nous avons le plaisir de vous présenter notre travail sur ce projet. Durant la deuxième année de la l2 Informatique
à l'Université d'Orléans, nous avons travaillé en groupe de quatre sur un projet de résolution de Taquin.
\end{abstract}
 
\newpage
\tableofcontents
\listoffigures
\newpage

\chapter{Introduction}
Nous allons ensemble, aborder quatres points principaux dans ce rapport:
\begin{itemize}
\item L'étude du projet
\item L'organisation de travail au sain du groupe
\item Le développement du code
\item L'analyse des algorithmes.
\end{itemize}
%
\section{Cahier des charges}

%
\chapter{Etude}
\par
L'étude du projet fut une première étape importante pour se mettre dans une bonne dynamique de groupe. Dès la première semaine nous avons réalisé une version jouable du Taquin, puis nous avons travaillé sur cette version durant toute la première phase d'essaie.
\par\noindent
Rapidement il nous a fallu faire des essaies sur ce jeu pour qu'il puisse se résoudre algorithmiquement. Nous avons chacun travaillé de notre coté sur nos idées pour pouvoir en tester un plus grand nombre, mais tout en restant en contact réguliérement pour avancer ensemble.
\par\noindent
Cette manière de répartir les tâches nous a permis de se rendre compte des difficultés que l'on allait rencontrer plus rapidement.
%
\section{Analyse de faisabilité}
\par
L'analyse de faisabilité fut un moment qui a duré du début du projet jusqu'aux vacances de février, durant cette phase chacun travaillant sur des fonctionnalités différentes, nous avons pu voir ou étaient les problèmes dans notre architecture de départ.
\par\noindent
Ces tests ont été trés divers. Nous avons travaillé sur les fonctionnalités VT100 du terminal et la récupération des touches tappées par l'utilisateur. Mais nous avons aussi créé des algorithmes pour essayer de résoudre le Taquin.
\par\noindent
Tous ces tests nous ont permis au mois de février d'avoir une idée claire de l'architecture du projet.
\section{Conception UML}
\par
Cette analyse de février a permis d'aboutir au diagramme UML suivant, diagramme qui n'a pas beaucoup évolué jusqu'à la version finale.
%INCLURE LE MODELE UML
\chapter{La gestion de projet}
%
\par
La gestion de projet est à la base de tout projet, et encore plus quand il se fait avec une équipe de quatre personnes.
\section{Le travail de groupe}
%
\par
La gestion de projet fut au coeur de nos préocupations avant même que le projet ne démarre,nous avons cherché à créer une équipe dynamique. Nous nous sommes donc vus régulièrement dans les salles de l'Université. Malgré celà il nous a fallu mettre en place des moyens dédiés pour faciliter le travail et éviter une dégradation de l'entente.
\par\noindent
 Dès la première semaine, nous avons mis en place un dépôt git sur le site de l'hébergeur GitHub, ce dépôt nous a permis d'avoir le réflexe de l'utiliser même  si nous avons rencontré quelques problèmes, il nous a permis a chacun de voir l'avancement du projet.
\par\noindent
 Cependant un tel dépôt ne répond pas à la question de la communication, chacun habitant une ville différente, nous avons donc communiquer via Skype pour remédier à ce problème.
\section{Le diagramme de Gantt}
%
\par
Nous avons utilisé la technique du diagramme de GANTT pour vérifier en interne que le projet avançait convenablement. Ci-dessous vous pouvez voir ce
diagramme.
\monimage{DiagrammeDeGantt.pdf}{Diagramme de GANTT}{DG}
\chapter{Phase de développement}
%
La phase de développement...
\section{L'architecture}
On explique les choix qu'on a fais
\section{Le codage}
%
\section{Junit et Benchmark}
%
\subsection{Utilisation des benchmarks}
\par
Nous avons utilisé les benchmarks pour améliorer l'efficacité de l'algorithme de solution. En effet, nous avons remarqué entre autres que la taille de l'ensemble incomplet avait une importance dans le temps de résolution. Nous avons donc créé un test avec plusieurs tailles d'ensemble pour voir laquelle permettait une resolution rapide.
\monimage{ensembleincomplettest2.pdf}{Exemple}{EX}
\chapter{Analyse et Conclusion}
%
\section{Analyse des Benchmark}
%
\subsection{Jeu simple}
%
\subsection{Jeu moyen}
%
\subsection{Jeu compliqué}
%
\section{Conclusion}
%
\newpage
\bibliography{bibliographie}{}
\bibliographystyle{plain}

\chapter{Resources utilisées}
\begin{itemize}
 \item JUNIT
 \item JAVA
 \item Dia
 \item Eclipse
 \item Argouml
 \item \LaTeX (sous kile)
 \item Google Drive
 \item Kate
 \item Git, GitHub
\end{itemize}

\end{document}
