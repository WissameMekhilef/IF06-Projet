\documentclass[a4paper,twoside,12pt]{report}
\usepackage[utf8x]{inputenc}
\usepackage{color}
\usepackage{fancyvrb} % pour mettre les verbatim dans des boites
%\usepackage[dvipdf]{graphics}
\usepackage[pdftex]{graphicx}
\usepackage{fancyhdr}
\usepackage{listings}
\usepackage{float}
\usepackage{listingsutf8}
\usepackage[T1]{fontenc}
\usepackage[french]{babel}
\usepackage{textcomp}
\usepackage[toc,page]{appendix}
%\usepackage{hyperref}
% Quelques réglages particuliers
\oddsidemargin=-0.1in %xx
\evensidemargin=-0.1in
\textwidth=6.1in
\topmargin=-0.5in
\textheight=8.7in
\parskip 0.25in
\lstset{
language=Java,
basicstyle=\normalsize,
upquote=true,
aboveskip={1.5\baselineskip},
columns=fullflexible,
showstringspaces=false,
extendedchars=true,
breaklines=true,
showtabs=false,
showspaces=false,
showstringspaces=false,
identifierstyle=\ttfamily,
keywordstyle=\color[rgb]{1,0,1},
commentstyle=\color[rgb]{0.133,0.545,0.133},
stringstyle=\color[rgb]{0.627,0.126,0.941},
}
\fancyhead[RO,RE]{\includegraphics[width=0.5in]{Logo-univ-orleans.png}}
\fancyhead[CO,CE]{Rapport de projet}
\fancyhead[LO,LE]{Jeu de Taquin}
\fancyfoot[C]{2014}
\fancyfoot[RO, LE] {\thepage}
%\fancyfoot[LO, RE] { ARNOULT S., MEKHILEF W., OUSSAD J., RETY M.}
%
%Quelques macros
\newcommand{\moncode}[1]{\begin{center}
                        \lstinputlisting[inputencoding=utf8/latin1]{#1}
                        \end{center}}

\newcommand{\monimage}[3]{
\par\noindent
\begin{figure}[H] %on ouvre l'environnement figure
\begin{center}
\includegraphics[width=0.7\textwidth]{#1} %ou image.png, .jpeg etc.
\caption{#2} %la légende
\label{#3} %l'étiquette pour faire référence à cette image
\end{center}
\end{figure} %on ferme l'environnement figure
}

\newcommand{\ml}[0]{\par\noindent}

%\hyphenation{supplémentaires}{sup-pl\'e-men-tai-res}
%opening
\title{\textcolor{blue}{\Large Rapport du projet IF06}\\\textcolor{blue}{\Large Jeu de Taquin}}
\vskip 1cm
\author{ARNOULT Simon, MEKHILEF Wissame, OUSSAD Jihad, RETY Martin \date{\today}}


\begin{document}

\thispagestyle{empty}
%
\begin{figure}[H]
\includegraphics[width=0.2\linewidth]{Logo-univ-orleans.png}
% \hfill

\end{figure}
\vspace{2cm}
%
\begin{center}
{\Huge Licence 2 Informatique\\\ \\Rapport du projet IF04}
\par\vspace{1.4cm}

{\Huge\bf \textcolor{red}{\bf Jeu de Taquin}}
\par\vspace{1.6cm}

{\Large       Réalisé par:}
\par\vspace{1.3cm}
{\large\bf \textcolor{blue}{ARNOULT Simon, MEKHILEF Wissame, OUSSAD Jihad, RETY Martin}}
\vfill
\today
\end{center}
\newpage
\pagestyle{fancy}

\begin{abstract}
%
\end{abstract}
 
\newpage
\tableofcontents
\listoffigures
\newpage

\chapter{Introduction}
% 
\section{Cahier des charges}
%
\chapter{Etude}
%
\section{Analyse de faisabilité}
Explication des différents test effectué jusqu'au vacances de février 
\section{Conception UML}
include d'un diagramme UML 
\chapter{La gestion de projet}
%
\section{Le travail de groupe}
%
Le travail de groupe étant un point primordial de ce projet, la communication entres les différents membres de l'équipe
à nécessité l'utilisation d'outils dédiée. Pour le partage du code nous avons privilégié l'utilisation de git et plus 
plus précisément du service fourni par la société GitHub qui permet l'hébergement d'un repository en ligne, chacun
suivre l'évolution du projet facilement. Pour la communication à distance skype à permis de pouvoir partager sur le
projet.
\section{Le diagramme de Gantt}
%
Nous avons utilisé la technique du diagramme de GANTT pour vérifier en interne que le projet avancé convenablement. Ci-dessous vous pouvez voir ce
diagramme.
\monimage{DiagrammeDeGantt.pdf}{Diagramme de GANTT}{DG}
\chapter{Phase de développement}
%
\section{L'architecture}
On explique les choix qu'on a fais
\section{Le codage}
%
\section{Junit et Benchmark}
%
\subsection{Utilisation des benchmarks}
On a utilisé les benchmark pour améliorer l'efficacité de l'algorithme de solution, en effet on a remarquer entre autre que la taille de l'ensemble
incomplet avais une importance dans le temps de résolution. On a donc créer un test avec plusieurs taille d'ensemble pour voir laquelle 
permetté une resolution rapide.
\monimage{ensembleincomplettest2.pdf}{Exemple}{EX}
\chapter{Analyse et Conclusion}
%
\section{Analyse des Benchmark}
%
\subsection{Jeu simple}
%
\subsection{Jeu moyen}
%
\subsection{Jeu compliqué}
%
\section{Conclusion}
%
\newpage
\bibliography{bibliographie}{}
\bibliographystyle{plain}

\chapter{Resources utilisées}
\begin{itemize}
 \item JUNIT
 \item JAVA
 \item Dia
 \item Eclipse
 \item Argouml
 \item \LaTeX (sous kile)
 \item Google Drive
 \item Kate
 \item Git, GitHub
\end{itemize}

\end{document}
